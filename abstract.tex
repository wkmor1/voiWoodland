$\dagger$ Corresponding author: wkmor1@gmail.com\\
1. School of BioSciences, the University of Melbourne, Parkville 3010, Victoria.\\
2. Patuxent Wildlife Research Center, US Geological Survey, Laurel, MD 20708.

\newpage

\section*{Abstract}\label{abstract}
\addcontentsline{toc}{section}{Abstract}

Value of information (VOI) analyses reveal the expected benefit of reducing uncertainty to a decision maker. Most ecological VOI analyses have focused on population models rarely addressing more complex community models. We performed a VOI analyses for a complex state and transition model of Box-Ironbark Forest and Woodland management. With three management alternatives (limited harvest/firewood removal, ecological thinning and no management), managing the system optimally (for 150 years) with the original information, would on average, increase the amount of forest in a desirable state from 19 to 35\% (a 16 percentage point increase). Resolving all uncertainty would, on average, increase the final percentage to 42\% (a 19 percentage point increase). However, only resolving the uncertainty for a single parameter was worth almost two-thirds the value of resolving all uncertainty. We found the VOI to depend on the number of management options, increasing as the management flexibility increased. Our analyses show it is more cost-effective to monitor low-density regrowth forest than other states, and more cost-effective to experiment with the no management alternative than the other management alternatives. Importantly, the most cost-effective strategies did not include either the most desired forest states, nor the least understood management strategy, ecological thinning. This implies that managers cannot just rely on intuition to tell them where the most value of information will lie, as critical uncertainties in a complex system are sometimes cryptic.

Keywords: Box-Ironbark, decision theory, monitoring, multivariate adaptive regression splines, optimization.

\documentclass[]{article}
\usepackage{lmodern}
\usepackage{amssymb,amsmath}
\usepackage{ifxetex,ifluatex}
\usepackage{fixltx2e} % provides \textsubscript
\ifnum 0\ifxetex 1\fi\ifluatex 1\fi=0 % if pdftex
  \usepackage[T1]{fontenc}
  \usepackage[utf8]{inputenc}
\else % if luatex or xelatex
  \ifxetex
    \usepackage{mathspec}
  \else
    \usepackage{fontspec}
  \fi
  \defaultfontfeatures{Ligatures=TeX,Scale=MatchLowercase}
  \newcommand{\euro}{€}
\fi
% use upquote if available, for straight quotes in verbatim environments
\IfFileExists{upquote.sty}{\usepackage{upquote}}{}
% use microtype if available
\IfFileExists{microtype.sty}{%
\usepackage{microtype}
\UseMicrotypeSet[protrusion]{basicmath} % disable protrusion for tt fonts
}{}
\usepackage[margin=1in]{geometry}
\usepackage{hyperref}
\PassOptionsToPackage{usenames,dvipsnames}{color} % color is loaded by hyperref
\hypersetup{unicode=true,
            pdftitle={The value of information for woodland management: updating a state and transition model},
            pdfauthor={William K. Morris, Michael C. Runge \& Peter A. Vesk},
            pdfborder={0 0 0},
            breaklinks=true}
\urlstyle{same}  % don't use monospace font for urls
\usepackage{natbib}
\bibliographystyle{apalike}
\usepackage{graphicx,grffile}
\makeatletter
\def\maxwidth{\ifdim\Gin@nat@width>\linewidth\linewidth\else\Gin@nat@width\fi}
\def\maxheight{\ifdim\Gin@nat@height>\textheight\textheight\else\Gin@nat@height\fi}
\makeatother
% Scale images if necessary, so that they will not overflow the page
% margins by default, and it is still possible to overwrite the defaults
% using explicit options in \includegraphics[width, height, ...]{}
\setkeys{Gin}{width=\maxwidth,height=\maxheight,keepaspectratio}
\setlength{\parindent}{0pt}
\setlength{\parskip}{6pt plus 2pt minus 1pt}
\setlength{\emergencystretch}{3em}  % prevent overfull lines
\providecommand{\tightlist}{%
  \setlength{\itemsep}{0pt}\setlength{\parskip}{0pt}}
\setcounter{secnumdepth}{0}

%%% Use protect on footnotes to avoid problems with footnotes in titles
\let\rmarkdownfootnote\footnote%
\def\footnote{\protect\rmarkdownfootnote}

%%% Change title format to be more compact
\usepackage{titling}

% Create subtitle command for use in maketitle
\newcommand{\subtitle}[1]{
  \posttitle{
    \begin{center}\large#1\end{center}
    }
}

\setlength{\droptitle}{-2em}
  \title{The value of information for woodland management: updating a state and
transition model}
  \pretitle{\vspace{\droptitle}\centering\huge}
  \posttitle{\par}
  \author{William K. Morris, Michael C. Runge \& Peter A. Vesk}
  \preauthor{\centering\large\emph}
  \postauthor{\par}
  \predate{\centering\large\emph}
  \postdate{\par}
  \date{2016-04-22}


\linespread{2}\selectfont
\usepackage{setspace}
\usepackage[labelsep = none, textformat = empty]{caption}
\usepackage[nomarkers]{endfloat}
\AtBeginDelayedFloats{\linespread{2}}
\usepackage{lineno}
\linenumbers

% Redefines (sub)paragraphs to behave more like sections
\ifx\paragraph\undefined\else
\let\oldparagraph\paragraph
\renewcommand{\paragraph}[1]{\oldparagraph{#1}\mbox{}}
\fi
\ifx\subparagraph\undefined\else
\let\oldsubparagraph\subparagraph
\renewcommand{\subparagraph}[1]{\oldsubparagraph{#1}\mbox{}}
\fi

\begin{document}
\maketitle

\newpage

\section{Abstract}\label{abstract}

Value of information (VOI) analyses reveal the expected benefit of
reducing uncertainty. Most ecological VOI analyses focus on population
ecology and discrete expressions of uncertainty, rarely addressing
complex community models or models with continuous uncertainty. We
performed VOI analyses for a complex state and transition model of
Box-Ironbark Forest and Woodland management. In contrast to previous
work, this study focuses on a community ecology-based decision, a
subfield of applied ecology where VOI has not been implemented before.
We also extend the field by demonstrating VOI for a decision model with
continuous uncertainty. Resolving all uncertainty in a system model of
Box-Ironbark management would improve the expected outcome. With three
management alternatives (limited harvest/firewood removal, ecological
thinning and no management), managing the system optimally (for 150
years) with the original information, would on average, increase the
amount of forest in a desirable state from 19 to 35\%. Resolving all
uncertainty would, on average, increase the final percentage to 42\%. In
other words, after resolving all uncertainty, we would expect to more
than double the performance from management. However, only resolving the
uncertainty for a single parameter was worth almost two-thirds the value
of resolving all uncertainty. We found the VOI to be dependent on the
number of management options. When the number of management options
increased, the value of perfect information increased linearly. However
the partial value of perfect information increased at different rates
for different model parameters. Our analyses show it is more
cost-effective to monitor low-density regrowth forest than other states,
and more cost-effective to experiment with the `no management'
alternative than the other management alternatives. Importantly, the
most cost-effective strategies did not include either the most desired
forest states, nor the least understood management strategy, ecological
thinning. This implies that managers cannot just rely on intuition to
tell them where the most value of information will lie, as critical
uncertainties in a complex system are sometimes cryptic.

Keywords: Box-Ironbark, multiplicative adaptive regression splines,
decision theory, monitoring, optimization. \newpage

\section{Introduction}\label{introduction}

Ecosystems are typically managed under uncertainty. Monitoring can
reduce uncertainty and facilitate management decisions with greater
expected benefits. On the other hand, sometimes it is unclear whether
monitoring is necessary \citep{Mcdonaldmadden2010}. The decision to
experiment or monitor at all is only rational if the expected benefit
when making a decision with new information, outweighs the cost of
learning. Determining the value of information facilitates calculating
this benefit \citep{Raiffa1968}.

Value of information (VOI) theory is a set of decision theoretic tools
that have previously been applied to decision problems in economics,
medicine, engineering and other domains
\citep{Dakins1999, Yokota2004, Claxton2008, Wu2013}. With the VOI
toolset, a decision analyst can assess what to monitor, how much to
monitor and even whether monitoring is justified at all. Compared with
other fields, such as operations research and medicine, ecology and
natural resource management have been slower to adopt VOI analyses for
decision making, although recently, some examples have appeared in the
literature
\citep[e.g.,][]{Polasky2001, Moore2011, Runge2011, Runting2013}. Most
examples have focused on the application of VOI to models with discrete
expressions of uncertainty \citep[e.g.,][]{Moore2012}. Furthermore,
ecological decision problem solvers have tended to focus on models of
population dynamics
\citep{Runge2011, Canessa2015, Johnson2014, Maxwell2015} and have been
less inclined to tackle more complex ecosystem level models.

One reason communities and ecosystems have rarely been the focus of VOI
analyses, is that community or ecosystem models do not lend themselves
easily to VOI analysis. Ecosystem models, such as state and transition
models, are complex and highly parameterized, and often models of
communities are continuous rather than discrete \citep[i.e., uncertainty
lies in the choices between discrete models as in][]{Runge2011} making
it more difficult to calculate the VOI analytically. Instead, VOI is
typically estimated using numerical methods \citep{Yokota2004}.

\subsection{Calculating the value of
information}\label{calculating-the-value-of-information}

A value of information analysis can be used to assess the benefit of
reducing epistemic uncertainty before making a decision. A decision
analyst cannot know, in advance, what information they will gain if they
seek to learn before taking action. With this in mind, the value of
information is conceptualized in terms of expected value. An expected
value is a value a decision analyst may expect, on average, given the
outcome of their decision is uncertain. Expected value is value weighted
(multiplied) by probability. For example, for two equally likely events
resulting in a value of one and two dollars respectively, the expected
value will be one and half dollars.

In this way a decision analyst can calculate the expected value of
information. In essence, the expected value \emph{of} information (EVI)
is the difference between the expected value with \emph{new} information
(EVWNI) and the expected value with \emph{original} information (EVWOI).
The expected value of information can come in a number forms depending
on the form of new information. Regardless of form, all the variants of
VOI analyses assume decisions are being made optimally. That is, the
decision analyst considers value to arise from taking actions that
maximize the their expected value given the original or new information.
To do so, the decision analyst requires a model that can be used to
predict the outcomes of possible future decisions. In calculating VOI
the analyst will use such a model to predict decision outcomes using the
original or new information as their model inputs.

As noted above, value of information can take multiple forms. Here we
deal we two forms: the expected value of perfect information (EVPI) and
the partial expected value of perfect information (EVPXI). We refer to
these and other related methods collectively, as EVI. Here we will
briefly outline the general form of an EVI analysis in terms of EVPI and
EVPXI. For a more detailed description of the derivation of EVI and its
variants, see the seminal text of \citet{Raiffa1968} and for more recent
treatments see \citet{Yokota2004} and \citet{Williams2011}.

The expected value of perfect information (EVPI) quantifies the expected
performance gain if all uncertainty is resolved prior to taking action
\citep{Raiffa1968}. The EVPI is the upper bound of expected performance
improvement and can identify the amount of resources worth investing in
to resolve uncertainty \citep{Runge2011}. While EVPI provides a value
for complete reduction of uncertainty, EVPXI, can quantify the
performance gain if uncertainty is only partially resolved
\citep{Ades2004}. EVPXI is the value of knowing the exact value of one
more (but not all) parameters in a model.

Again, the EVI is the difference between the EVWNI and the EVWOI. When
the new information is perfect (i.e., the analysts model will have all
uncertainty eliminated) then it follows that:

\[ \mathrm{EVPI} = \mathrm{EVWPI} - \mathrm{EVWOI}, \]

where EVWPI is the expected value \emph{with} perfect information.

To calculate the expected values EVWPI and EVWOI, like any expected
value, the analyst will work out the value they expect to see on average
when taking the most optimal, or maximizing actions. More formally:

\[ \mathrm{EVWOI} = Max_{a}[Mean_{s}(Value)] \]

and

\[ \mathrm{EVWPI} = Mean_{s}[Max_{a}(Value)]. \]

Where \(a\) indicates the actions available and \(s\) indicates initial
the uncertainty or the state space (the world of possible scenarios that
could lead an action to generate some value). As you can see, both the
equations for EVWPI and EVWOI are similar. The key difference is the
order of maximization (optimizing) and averaging (taking the mean over
the uncertainty). To calculate EVWOI (working from the inside of the
equation out) first average over the initial uncertainty and then take
the action which maximizes value. Whereas for the EVWPI the opposite is
true. First choose the action that maximize value and only then, average
this value over the initial uncertainty. Calculating the EVPXI requires
a similar yet more complicated approach. For a model with multiple
parameters to calculate the EVPXI for the \(i^{th}\) parameter(s) of
interest is expressed formally as:

\[ \mathrm{EVPXI_i} = \mathrm{EVWPXI_i} - \mathrm{EVWOI}, \]

where,

\[ \mathrm{EVWPXI_i} = Mean_{s_i}\{Max_{a}[Mean_{s_c}(Value)]\}. \]

Here \(c\) represents the rest of the parameters in the model with
uncertainty, \(s\). Partial perfect information requires an additional
`averaging over' step, where value is averaged over the \(c^{th}\)
not-of-interest parameters, then maximization occurs, before finally
averaging over the initial uncertainty of the parameter(s) of interest
takes place.

In the following work we apply the above calculations of EVPI and EVPXI
to a case study on Box Ironbark Forests and Woodland to ascertain
whether and how the addition of new information may improve the outcome
of their management.

\subsection{Box Ironbark Forest and Woodland
management}\label{box-ironbark-forest-and-woodland-management}

The Box Ironbark Forest and Woodland (BIFAW) region covers approximately
250,000ha of central Victoria, Australia. The BIFAW are dry plant
communities that occur on low-fertility soils and in a semi-arid to
temperate climate. Much of the pre-European stands of BIFAW were cleared
for agriculture and gold mining. Most of the current extent is highly
fragmented regrowth. These regrowth stands are typically missing key
ecosystem components such as large, hollow-bearing trees and a diverse
understory shrub layer. Tree species found in BIFAW include Grey Box
(\emph{Eucalyptus microcarpa}), Red Box (\emph{E. polyanthemos}), Long
Leaf Box (\emph{E. goniocalyx}), Yellow Box (\emph{E. melliodora}), Red
Ironbark (\emph{E. tricarpa}), Red Stringybark (\emph{E. macrorhyncha})
and Yellow Gum (\emph{E. leucoxylon}). The BIFAW supports important
habitat for three Victorian listed threatened taxa: the Brush-tailed
Phascogale (\emph{Phascogale tapoatafa}), Powerful Owl (\emph{Ninox
strenua}) and Regent Honeyeater (\emph{Xanthomyza phrygia})
\citep{Tzaros2005}. The latter is also a federally listed endangered
species. Regent Honeyeaters are threatened because they rely on large
trees for foraging and nesting \citep{Menkhorst1999}.

In 1996 the state government of Victoria commissioned an investigation
into an appropriate system for the protection and management of BIFAW.
As a direct result, over 200,000 ha of BIFAW were gazetted as national
park and other protected areas. A key recommendation of the report was
that `ecological thinning' be part of the management strategy of the
BIFAW \citep{Pigott2009}. The report recommended a program of ecological
thinning be undertaken as part of an ecological management strategy to
assist the development of a forest structure ultimately dominated by
large diameter trees in the new parks and reserves \citep{ECC2001}.
`Ecological thinning' is the active reduction of stem density to improve
forest health, while retaining some fallen timber to improve habitat for
plants and animals \citep{Cunningham2009}. In 2003 the body in charge of
BIFAW park management, Parks Victoria, established an `ecological
thinning trial'. The aim of the trial was to investigate whether
`ecological thinning' (hereby `thinning') could be used to restore
structural diversity of habitat types and the functioning and
persistence of key communities and species populations
\citep{Pigott2010}.

\subsection{State and transition models of Box Ironbark Forest and
Woodland}\label{state-and-transition-models-of-box-ironbark-forest-and-woodland}

\citet{Czembor2009} built a suite of simulation models of BIFAW
dynamics, parameterized through expert elicitation using methods adapted
from \citet{Morgan1990}. Their BIFAW state and transition models predict
the proportions of a model landscape in four states, including the
states most desired by land-managers, low-density mature woodland and
high-density mature woodland. The models included three management
scenarios, no management/natural disturbance only (NM), a scenario of
limited harvest/firewood removal (HF), and thinning (ET). In analyzing
the variation among models which included three components of
uncertainty, \citet{Czembor2011} found that between-expert uncertainty
was the greatest contributor to total model uncertainty, followed by
imperfect individual expert knowledge, then system stochasticity. In the
case presented here we treated the opinions of a set of experts as the
initial state of uncertainty.

\subsection{Study objectives}\label{study-objectives}

Management of uncertain systems such as the BIFAW can benefit from
investment in learning about the system itself. But in complex systems
with many parameters and multiple state variables changing at different
rates, which aspects of the system should we learn about? To answer this
question we undertook an EVI analysis using the models of
\citet{Czembor2011}. To find the upper bound on the value of information
and see if there was any real value in new information we calculated
EVPI. We then calculated the EVPXI for all transition probabilities to
determine which parameters have the greatest value for learning. We also
analysed a set of sampling strategies that reduce uncertainty about
multiple parameters at once by calculating EVPXI for two parameters of
transitions out of each system state and two parameters of transitions
for each management strategy. Finally we tested the effect of varying a
constraint on action to see what happened to the EVI when there were
more or less options available to a manager. \newpage

\section{Methods}\label{methods}

\subsection{Decision problem}\label{decision-problem}

The fundamental objective of the BIFAW region managers are to ensure the
persistence of key, functioning communities and species populations of
the region. Managers have determined that the means of achieving these
objectives is to maximize the proportion of the landscape in a mature
woodland (either high or low-density) state which supports key habitat
components such as hollow bearing trees and a diverse shrubby
understory. They have three management actions available to achieve this
objective: no management/allow natural disturbance only, allow limited
harvest/firewood removal and thinning.

Therefore, for the following analyses, the optimization steps have the
objective of maximizing the proportion of the BIFAW in either
low-density or high-density mature woodland. Also, for the analyses that
follow, we place the constraint that only 20\% of the BIFAW can be
`thinned' (managers use the ET option) as a larger proportion seems
infeasible. However, we also test this constraint by varying the
allowable proportion of thinning from 10 to 100\% (see below).

The case study, as outlined here, falls in class of problems known as
linear optimizations, as such the solutions to the objective
maximization will always lie on one of the vertices of the feasible
region of management. In other words, the action that maximizes the
objective will be be either 100\% NM, 100\% HF, 20\% ET and 80\% NM or
20\% ET and 80\% HF (or the equivalent proportions when the constraint
on ET is different; Figure 1.).

\subsection{Predicting the outcome of management under
uncertainty}\label{predicting-the-outcome-of-management-under-uncertainty}

The system model we used to make predictions of BIFAW management
decision outcomes were multivariate adaptive regression splines (MARS)
\citep{Friedman1991}, a regression modelling technique, fit to the
output data from the state and transition models of \citet{Czembor2011}.
We chose to represent the state and transitions using MARS rather than
run further state and transition models as the latter would be
computationally infeasible given the large number repeated model runs
needed for the analyses.

The state and transition models predicted the proportion of the modeled
BIFAW landscape in four vegetation states after 150 years. The four
states in the models were high-density regrowth (HDRG), low-density
regrowth (LDRG), high-density mature woodland (HDMW), and low-density
mature woodland (MLDW). The transition probabilities among these states
for each of the three management actions were parameterized by five
experts \citep{Czembor2009}. Transition probabilities were elicited for
a set of causal agents. The set of transition probabilities was
different for each of the three management scenario \citep[see Table 1
in][]{Czembor2009}. Combining transition types, management scenarios,
and causal agents, there were 169 different transition probability
parameters elicited from each of the five experts. There were other
parameter types included in the state and transition models, but for
simplicity we focus solely on the transition probabilities.

The expert elicitation resulted in 375 separate estimates of each
transition probability. And it is the distribution of these 375
estimates that constitutes the initial uncertainty of the decision model
in this case study. These estimates can be thought of as 375 alternative
scenarios for the trajectory of the BIFAW over the next 150 years. This
is much like a climate modelling may produce multiple alternative
trajectories of the climate into the future under different warming
scenarios.

\citet{Czembor2011} used the 375 as alternative parameterizations to
simulate BIFAW forest dynamics with the state and transition modelling
software package, Vegetation Dynamics Development Tool (VDDT)
\citep{ESSA2007} running the model ten times for each scenario. Here, we
took the output from their simulations and fit MARS models for each
management action separately. For each of the MARS models the response
(\(n=375\times10=3750\)) was the proportion of the model landscape in
the two mature woodland states combined. The predictors were the model
parameters (including but not limited to the transition probabilities).

All analyses were done using the statistical computing language R
version 3.2.4 \citep{R2015} with the MARS models fit using the software
package `earth' version 4.4.4 \citep{Milborrow2013}.

\subsection{Calculating EVI}\label{calculating-evi}

\subsubsection{EVPI}\label{evpi}

To calculate we applied equation 1, 2, and 3 to the outcomes of BIFAW
management predicted by MARS models for each of the 375 alternative
expert-derived parameterizations (which is same as the original
simulated output of \citet{Czembor2011} averaged over the 100 model
runs). These data can be transformed into a 4 by 375 matrix where cells
hold the predicted proportion of the landscape in a mature state, with
the rows representing alternative input parameter sets and columns
representing the potentially optimal solutions as in Figure 1. This
matrix can then be used to both calculate EVWOI and EVWPI and by
extension EVPI.

To calculate the EVWOI we first average across the rows (down the
columns) to ascertain the expected value of each potentially optimal
solution. Then choose the solution with the maximum value and this will
be the EVWOI. Calculating EVWPI requires the opposite approach. First we
work with each row individually and choose the column that maximizes the
value as if we new with certainty that the particular parameter set
associated with a given row was correct. After we have maximized the
value of each of the 375 alternative scenarios only then do we take the
average of these which will be the EVWPI. Again the EVPI is simply the
difference between these values.

\subsubsection{EVPXI}\label{evpxi}

EVPXI requires a slightly more complicated algorithm. The EVWOI in
equation 4, which is used to calculate EVPXI can be calculated as above.
However, calculating the EVWPXI for the \(i^{th}\) parameter(s) requires
the algorithm which we outline in pseudo-code in Box 1.

\begin{center}\rule{0.5\linewidth}{\linethickness}\end{center}

\paragraph{\texorpdfstring{Box 1: Calculating
EVWPXI\(_i\)}{Box 1: Calculating EVWPXI\_i}}\label{box-1-calculating-evwpxiux5fi}

\textbf{For each parameter value,} \(p\) from 1 to 375

\(\quad\) Step 1. Set parameter(s) \(i\) to parameter value \(p\).

\(\quad\) \textbf{For each parameter value,} \(p'\) from 1 to 375

\(\quad\) \(\quad\) Step 2. Set parameters \(c\) to parameter value
\(p'\).

\(\quad\) \(\quad\) Step 3. Predict the proportion of BIFAW in a mature
state with the parameters set in steps 1 and 2.

\(\quad\) \(\quad\) Step 4. Record the value of each potentially optimal
management strategy : \(Value\).

\(\quad\) \textbf{end loop}

\(\quad\) Step 5. Average the result of step 4 across each parameter
value \(p'\) : \(Mean_{s_c}(Value)\).

\(\quad\) Step 6. Record the maximum expected value from step 5 :
\(Max_a[Mean_{s_c}(Value)]\).

\textbf{end loop}

Average the result of step 6 across each parameter value \(p\) :
\(Mean_{s_i}\{Max_a[Mean_{s_c}(Value)]\}\).

\begin{center}\rule{0.5\linewidth}{\linethickness}\end{center}

Using the algorithm above we calculated the EVPXI for all 169 transition
probabilities. Using the same algorithm we then calculated the EVPXI for
pairs of parameters simultaneously. First we took the top two most
valuable transition probabilities that transition away from each of the
four woodland states and calculated their simultaneous EVPXI. Then we
did the same for the top pair of of parameters for each management
strategy. With these results we can see which state or management
strategy it would be most beneficial to focus on for learning.

Finally we recalculated the EVPI and EVPXI for all transition
probabilities this time varying the constraint on the amount of ET
management allowable. This has the effect of changing the size of the
feasible management region and changing in the position of the two
left-most vertices in Figure 1. We recalculated EVI for a maximum
allowable rate of thinning from 10 to 100\% in increments of 10\%.
\newpage

\section{Results}\label{results}

\subsection{EVWOI}\label{evwoi}

With the original information elicited from the five experts, the
optimal decision would be to put 100\% of the BIFAW under the NM option
regardless of the constraint applied to the ET option. Given the optimal
decision is made with the original information, managers would expect,
on average, to see approximately 35\% of the BIFAW in a mature state
after 150 years. This represents a land area of 88,462 ha, a 40,962 ha
increase over the initial amount (47,500 ha, 19\%) estimated by the five
experts.

\subsection{EVPI}\label{evpi-1}

If the managers of the BIFAW had perfect knowledge and could at most,
thin at rate of 20\%, they would expect on average to see 42\% mature
woodland after 150 years (which is an EVWPI of 104,276 ha) this means
that the EVPI is 6\% or 15,814 ha (Figure 2).

\subsection{EVPXI}\label{evpxi-1}

Of the 169 transition probabilities most, but not all, had zero or
negligible EVPXI (i.e., EVWXI\(_i\) \(\approx\) EVWOI). Figure 1 (top
panel) shows the top four most valuable parameters, while remaining
parameters had EVPI of \%1 or less, notably, one parameter (the
probability of woodland transitioning from a low-density to a high
density regrowth state, due coppicing of tree stems, when the BIFAW is
left unmanaged) was 4\%. To managers, this means if they could
completely resolve the uncertainty in this parameter alone they would
expect to manage the BIFAW so much more optimally that on average they
would see 9,857 ha more mature woodland in 150 years than if they
managed the forest under their initial level of uncertainty.

\subsection{EVI when constraint on management
changes}\label{evi-when-constraint-on-management-changes}

Figure 3 shows the effect of changing the allowable amount of thinning
of the BIFAW from 10 to 100\% on EVPI and the most valuable parameter
(according to the EVPXI analysis above) for each management action. Note
that while both increase linearly as the constraint is lifted those
parameters unrelated to the ET remain the same despite the change in the
feasible region.

\newpage

\section{Discussion}\label{discussion}

The most important implication of these analyses for managers, is that
the most cost-effective strategies did not include either the most
desired states (mature low-density or high-density woodlands), nor what
would be thought of as the least understood management strategy
(thinning). As such, managers cannot just rely on intuition to tell them
where the most value of information is. Critical uncertainties (those
uncertainties that affect decision making) in a complex system might be
not be the most variable, or the uncertainties to which outcomes are
most sensitive.

We have identified those aspects of the BIFAW system model that have the
greatest value for learning. If monitoring was targeted at the most
valuable parameter according to the EVPXI, rather than a random
parameter or even the most uncertain parameter, then the expected
performance of a subsequent decision would be greater by up to 9,857
hectares of mature woodland given the number of management options and
constraint we used here, which is an average of 66 hectares per year.

We have also shown that if learning is targeted at subsets of the
system, so as to update multiple parameters simultaneously, then some
system states and management options would be better foci than others.
This is because the uncertainty of some parameter sets is more critical
to management decisions than others. Managers would be wiser to focus
monitoring on areas subject to management under the HF scenario rather
or ET or NM options, because learning by reducing the uncertainty in the
parameters associated with those latter options would not change the
decision about which management option to use as much on average if
their uncertainty was reduced. Also, if managers chose only to monitor
one system state, then learning about transitions from the low-density
regrowth state would see greater expected benefit than the other three
states.

However, the value of information depends on the number and range of
management options. In other words, the options available to manager can
drive the value of information. The greater the number and range of
management options, the greater the value of information, because the
more options you have the more potential there is for learning which
option is the best. Taken to an extreme, if you have only one option
(i.e., no decision to make) then learning would be pointless and the
value of information would be zero. In the present case, when more
thinning was permitted, EVPI and the EVPXI of the parameter predicting a
transition after thinning increased linearly whereas the EVPXI of
parameters not associated with thinning didn't change at all (Fig 3).

We found variability in the EVPXI for parameters of the BIFAW system.
Those with the greatest expected value of information were often those
with uncertainty to which the system was most sensitive, but not always.
Similarly, \citet{Moore2012} found that the main drivers of willow
invasion were not necessarily the same as those that there was most
value in learning about. In their case, while fire frequency was a
driver of invasion, it was willow tree seed dispersal that had the
greatest value of information. These results highlight that for optimal
decision making it is not enough to simply identify those aspects of the
system to which objectives are most sensitive. In the context of
decision making, the expected value of information is a better measure
of sensitivity than a simple sensitivity analysis because it can
distinguish between decision-critical uncertainty and mere uncertainty
\citep{Felli1998}.

A key advance we have made in this work is to formulate a process for
calculating VOI for complex models with continuous expressions of
uncertainty. For such models the state space is too large to make
analytical integration feasible for the VOI analyses and numerical
methods must be used instead. However, such an approach can be
computationally expensive models such as state and transition models as
it is prohibitively resource-consuming to refit a complex model
repeatedly. To overcome this, we represented the state and transition
model with an efficient regression method (MARS) and use the output from
this in our calculations of the EVI. The method we present can be used a
template for VOI analyses of complex models with continuous expressions
of uncertainty.

The analyses we performed required us to make assumptions regarding the
monitoring to update the parameters in the model. For monitoring to take
place in the manner described, vegetation states must be able to easily
be determined in the field. In addition they must be identifiable to a
degree of accuracy that they can be distinguished from one another so
that a transition between states is evident and recordable after
revisits occurring in a short space of time. States must not only
themselves be identifiable, but the surveyor must also be able to tell
that a unit of vegetation has been in a state for the required period of
time for a specific transition to occur, and that any transition that
does occur has occurred due to a specific causal agent. For some
transition probabilities this set of assumptions seems plausible. For
instance, the transition from low to high-density regrowth due to
coppicing may not be that hard to identify and record, whereas
recognizing that a unit of vegetation remained as low-density regrowth
because of combination of drought and mistletoe may be far more
difficult. This problem arises because state-transition models are only
supposed to represent vegetation dynamics on average and do necessarily
reflect observable phenomenon.

In summary we would recommend a value of information analyses be
performed to inform monitoring and even decide whether it should take
place at all. We have shown that it can be used to identify which parts
of a complex model system are most valuable to address. Here, given the
assumptions we have outlined above, managers should focus scarce
monitoring resources first on the harvest management option (if that
option is on the table) and the low-density regrowth states. Here we
have demonstrated how to overcome the challenges of implementing a VOI
analyses for a complex, continuous model. \newpage

\section{Acknowledgements}\label{acknowledgements}

We acknowledge Christina Czembor for conducting the initial research
upon which these analyses are based. We also thank Patrick Piggot and
Parks Victoria for their assistance. This work was supported by the
Australian Research Council through the Centre of Excellence in
Environmental Decisions and Linkage Project LP110100321; and the
Victorian Department of Sustainability and Environment. \newpage

\begin{figure}[htbp]
\centering
\includegraphics{voiWoodland_files/figure-latex/management_plot-1.pdf}
\caption{Ternary diagram highlighting the feasible region of managment
(maximum allowed thinning of 20\%) and the four vertices representing
potential optimal solutions to the decision problem.}
\end{figure}

\begin{figure}[htbp]
\centering
\includegraphics{voiWoodland_files/figure-latex/results_plot1-1.pdf}
\caption{The expected value of partial perfect information for the top
four transition probabilities on their own (top panel), the top two
transition probabilities out of each state (middle panel) and the top
two transition probabilities for each management option (bottom row)
compared with the expected value of perfect information (EVPI).}
\end{figure}

\begin{figure}[htbp]
\centering
\includegraphics{voiWoodland_files/figure-latex/results_plot2-1.pdf}
\caption{Response of the expected value information to changing the
constraint on the allowable proportion of the BIFAW to undergo
ecological thinning.}
\end{figure}

\renewcommand\refname{Literature cited}
\bibliography{voiWoodland.bib}

\end{document}
